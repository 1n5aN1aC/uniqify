\documentclass[letterpaper,10pt,titlepage]{article}

\usepackage{graphicx}                                        

\usepackage{amssymb}                                         
\usepackage{amsmath}                                         
\usepackage{amsthm}                                          

\usepackage{alltt}                                           
\usepackage{float}
\usepackage{color}

\usepackage{url}

\usepackage{balance}
\usepackage[TABBOTCAP, tight]{subfigure}
\usepackage{enumitem}

\usepackage{pstricks, pst-node}

\usepackage{geometry}
\geometry{textheight=9in, textwidth=6.5in}

%random comment

\newcommand{\cred}[1]{{\color{red}#1}}
\newcommand{\cblue}[1]{{\color{blue}#1}}

\usepackage{hyperref}

\def\name{Joshua Villwock}

%pull in the necessary preamble matter for pygments output
\input{pygments.tex}

%% The following metadata will show up in the PDF properties
\hypersetup{
  colorlinks = true,
  urlcolor = black,
  pdfauthor = {\name},
  pdfkeywords = {cs311 ``operating systems'' files filesystem I/O},
  pdftitle = {CS 311 Project 3: UNIX File I/O},
  pdfsubject = {CS 311 Project 3},
  pdfpagemode = UseNone
}

\parindent = 0.0 in
\parskip = 0.2 in

\begin{document}
\tableofcontents

\section{NOTICE}

This Project was sadly not finish due to explsions and time constraints.  Keep reading this document for full details.

The source code has been heavily commented, and while the code for doing the entire Project is at least 90% there, and compiles, it does pretty much nothing.

Also note that the code is largely in Main().  It would have been moved out to seperate functions once it was functioning properly.

Feel free to read the comments in the source code to learn how exactly everything was supposed to be done.

The rest of this document assumes the program worked as intended, except where it mentions something going wrong :)

\section{Overall System Design}

The Program will determin from the arguments how many worker thread should be created.  Will default to 10 if use does not specify.

The Main Proccess Will Fork off that many working proccesses (children).  It will then Proceed to reading CIN, after closing the unnecisary File Descriptors.

The Children Proccesses will Imediently Close all uneeded File descriptors, and then will exec() sort with no arguments, after changing its STDIN and STDOUT to the pipes, using dup2.

The main (distributor) proccess will read one line atr a time from STDIN, and break it apart, and then begin distributing it to each of the children (worker) threads in a round-robin fashion.

Once the entire STDIN has been read, and eof has been encountered, or the pipe closed, It will pass this on to the child threads, and close it's pipes, telling the sort() proccesses to begin.

Before this began, Another Proccess was forked off to cath the output of the child (worker) proccesses.  When it runs, it first closes the unused FD's in it, Then waits for anything coming though its pipes.

Once the child proccesses have finished sorting, the merger proccess will keep a "buffer" of the "next" element from each of the pipes.  It will then find the smallest (lowest) value in those buffers, and output it.

This proccess is described very well in the source code, along with how you would easioly implement the extra credit portion of the assignment.

\section{Work Log}

November 2nd: Parser largely done

Nov 4th: Code for Fork children, and work distribution, as well as begined merger code.

5th: tried to finish it up :/

Full work log can be found here:
\url{https://github.com/1n5aN1aC/uniqify/commits/master}

\section{Challenges}

See Question and answears for more info.

Debugging this Phantom bug described below.

Everything went pretty well, until I assumed It would play nice, where all of a sudden, code that was tested working before, decided not to work, the children refused to run properly, and everything fell apart.

\section{Questions And Answers}

\subsection{what do you think the main point of this assignment is?}

\begin{itemize}
\item To learn to use Proccesses Properly
\item To learn to use Pipes well.
\end{itemize}

\subsection{how did you ensure your solution was correct? Testing details, for instance.}

Much testing was done, but 99% of it was hunting a nearly-unreproducable phantom bug that caused the children to pretty much not work at all.  Even just a single Cout Statement, making sure that the STDout for the child was NOT overwritten with the pipe.

\subsection{what did you learn?}

Phantom bugs can spend a LOT of your time.

Even though you feel very confident with yourself because you get a lot done in one day, doesn't mean the rest of the project will go well.  It may implode, liek this one did.

\end{document}
